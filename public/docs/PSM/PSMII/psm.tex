\documentclass[]{article}
\usepackage{lmodern}
\usepackage{amssymb,amsmath}
\usepackage{ifxetex,ifluatex}
\usepackage{fixltx2e} % provides \textsubscript
\ifnum 0\ifxetex 1\fi\ifluatex 1\fi=0 % if pdftex
  \usepackage[T1]{fontenc}
  \usepackage[utf8]{inputenc}
\else % if luatex or xelatex
  \ifxetex
    \usepackage{mathspec}
  \else
    \usepackage{fontspec}
  \fi
  \defaultfontfeatures{Ligatures=TeX,Scale=MatchLowercase}
\fi
% use upquote if available, for straight quotes in verbatim environments
\IfFileExists{upquote.sty}{\usepackage{upquote}}{}
% use microtype if available
\IfFileExists{microtype.sty}{%
\usepackage[]{microtype}
\UseMicrotypeSet[protrusion]{basicmath} % disable protrusion for tt fonts
}{}
\PassOptionsToPackage{hyphens}{url} % url is loaded by hyperref
\usepackage[unicode=true]{hyperref}
\hypersetup{
            pdftitle={Propensity score methods-I PS Matching},
            pdfborder={0 0 0},
            breaklinks=true}
\urlstyle{same}  % don't use monospace font for urls
\usepackage[margin=1in]{geometry}
\usepackage{color}
\usepackage{fancyvrb}
\newcommand{\VerbBar}{|}
\newcommand{\VERB}{\Verb[commandchars=\\\{\}]}
\DefineVerbatimEnvironment{Highlighting}{Verbatim}{commandchars=\\\{\}}
% Add ',fontsize=\small' for more characters per line
\usepackage{framed}
\definecolor{shadecolor}{RGB}{248,248,248}
\newenvironment{Shaded}{\begin{snugshade}}{\end{snugshade}}
\newcommand{\KeywordTok}[1]{\textcolor[rgb]{0.13,0.29,0.53}{\textbf{#1}}}
\newcommand{\DataTypeTok}[1]{\textcolor[rgb]{0.13,0.29,0.53}{#1}}
\newcommand{\DecValTok}[1]{\textcolor[rgb]{0.00,0.00,0.81}{#1}}
\newcommand{\BaseNTok}[1]{\textcolor[rgb]{0.00,0.00,0.81}{#1}}
\newcommand{\FloatTok}[1]{\textcolor[rgb]{0.00,0.00,0.81}{#1}}
\newcommand{\ConstantTok}[1]{\textcolor[rgb]{0.00,0.00,0.00}{#1}}
\newcommand{\CharTok}[1]{\textcolor[rgb]{0.31,0.60,0.02}{#1}}
\newcommand{\SpecialCharTok}[1]{\textcolor[rgb]{0.00,0.00,0.00}{#1}}
\newcommand{\StringTok}[1]{\textcolor[rgb]{0.31,0.60,0.02}{#1}}
\newcommand{\VerbatimStringTok}[1]{\textcolor[rgb]{0.31,0.60,0.02}{#1}}
\newcommand{\SpecialStringTok}[1]{\textcolor[rgb]{0.31,0.60,0.02}{#1}}
\newcommand{\ImportTok}[1]{#1}
\newcommand{\CommentTok}[1]{\textcolor[rgb]{0.56,0.35,0.01}{\textit{#1}}}
\newcommand{\DocumentationTok}[1]{\textcolor[rgb]{0.56,0.35,0.01}{\textbf{\textit{#1}}}}
\newcommand{\AnnotationTok}[1]{\textcolor[rgb]{0.56,0.35,0.01}{\textbf{\textit{#1}}}}
\newcommand{\CommentVarTok}[1]{\textcolor[rgb]{0.56,0.35,0.01}{\textbf{\textit{#1}}}}
\newcommand{\OtherTok}[1]{\textcolor[rgb]{0.56,0.35,0.01}{#1}}
\newcommand{\FunctionTok}[1]{\textcolor[rgb]{0.00,0.00,0.00}{#1}}
\newcommand{\VariableTok}[1]{\textcolor[rgb]{0.00,0.00,0.00}{#1}}
\newcommand{\ControlFlowTok}[1]{\textcolor[rgb]{0.13,0.29,0.53}{\textbf{#1}}}
\newcommand{\OperatorTok}[1]{\textcolor[rgb]{0.81,0.36,0.00}{\textbf{#1}}}
\newcommand{\BuiltInTok}[1]{#1}
\newcommand{\ExtensionTok}[1]{#1}
\newcommand{\PreprocessorTok}[1]{\textcolor[rgb]{0.56,0.35,0.01}{\textit{#1}}}
\newcommand{\AttributeTok}[1]{\textcolor[rgb]{0.77,0.63,0.00}{#1}}
\newcommand{\RegionMarkerTok}[1]{#1}
\newcommand{\InformationTok}[1]{\textcolor[rgb]{0.56,0.35,0.01}{\textbf{\textit{#1}}}}
\newcommand{\WarningTok}[1]{\textcolor[rgb]{0.56,0.35,0.01}{\textbf{\textit{#1}}}}
\newcommand{\AlertTok}[1]{\textcolor[rgb]{0.94,0.16,0.16}{#1}}
\newcommand{\ErrorTok}[1]{\textcolor[rgb]{0.64,0.00,0.00}{\textbf{#1}}}
\newcommand{\NormalTok}[1]{#1}
\usepackage{graphicx,grffile}
\makeatletter
\def\maxwidth{\ifdim\Gin@nat@width>\linewidth\linewidth\else\Gin@nat@width\fi}
\def\maxheight{\ifdim\Gin@nat@height>\textheight\textheight\else\Gin@nat@height\fi}
\makeatother
% Scale images if necessary, so that they will not overflow the page
% margins by default, and it is still possible to overwrite the defaults
% using explicit options in \includegraphics[width, height, ...]{}
\setkeys{Gin}{width=\maxwidth,height=\maxheight,keepaspectratio}
\IfFileExists{parskip.sty}{%
\usepackage{parskip}
}{% else
\setlength{\parindent}{0pt}
\setlength{\parskip}{6pt plus 2pt minus 1pt}
}
\setlength{\emergencystretch}{3em}  % prevent overfull lines
\providecommand{\tightlist}{%
  \setlength{\itemsep}{0pt}\setlength{\parskip}{0pt}}
\setcounter{secnumdepth}{0}
% Redefines (sub)paragraphs to behave more like sections
\ifx\paragraph\undefined\else
\let\oldparagraph\paragraph
\renewcommand{\paragraph}[1]{\oldparagraph{#1}\mbox{}}
\fi
\ifx\subparagraph\undefined\else
\let\oldsubparagraph\subparagraph
\renewcommand{\subparagraph}[1]{\oldsubparagraph{#1}\mbox{}}
\fi

% set default figure placement to htbp
\makeatletter
\def\fps@figure{htbp}
\makeatother

\usepackage{etoolbox}
\makeatletter
\providecommand{\subtitle}[1]{% add subtitle to \maketitle
  \apptocmd{\@title}{\par {\large #1 \par}}{}{}
}
\makeatother
% https://github.com/rstudio/rmarkdown/issues/337
\let\rmarkdownfootnote\footnote%
\def\footnote{\protect\rmarkdownfootnote}

% https://github.com/rstudio/rmarkdown/pull/252
\usepackage{titling}
\setlength{\droptitle}{-2em}

\pretitle{\vspace{\droptitle}\centering\huge}
\posttitle{\par}

\preauthor{\centering\large\emph}
\postauthor{\par}

\predate{\centering\large\emph}
\postdate{\par}

\title{Propensity score methods-I PS Matching}
\date{}

\begin{document}
\maketitle

Propensity score methods include 1) propensity score matching; 2)
stratification on the propensity score; 3)IPTW: inverse probability of
treatment weighting; 4) covariate adjustment using the propensity score.
The codes below only apply to propensity score matching.

\begin{enumerate}
\def\labelenumi{\arabic{enumi}.}
\item
  Variables considering in the model: To correctly specify the
  propensity score model, one should include the true confounders (the
  risk factors that are associated with the `treatment' and outcome) and
  the potential confounders (the risk facotrs that are associated with
  the outcome). A safe way to do when sample size is not a concern is to
  include all baseline risk factors, but this will reduce the matching
  rate expecially when sample size is small.
\item
  To check balance of propensity score matched sample, one should use
  standardized difference. A threshold of less than 10\% of standardized
  difference may be used to consider balancing between treatment and
  control group. Using p-values to determine balance may be discouraged
  given that 1)p-values are sensitive to sample size. 2) p values are
  inference from a ``superpopulation'', the balance is an attribute to
  the particular matched sample. One could calculate the standardized
  difference with the following codes, or use the following formula.
\end{enumerate}

The standardized difference for a continous variable is

\[ d= \frac {\overline{x}_{treatment}-\overline{x}_{control}}{\sqrt {\frac{s^2_{treatment}+s^2_{control}}{2}}} \]

The standardized difference for a binary variable is

\[ d=\frac {\hat{p}_{treatment}-\hat{p}_{control}}{\sqrt {\frac{\hat{p}_{treatment} (1-\hat{p}_{treatment})+ \hat{p}_{control} (1-\hat{p}_{control})}{2}}} \]

Reference: Peter C. Austin.An Introduction to Propensity Score Methods
for Reducing the Effects of Confounding in Observational Studies.

\begin{Shaded}
\begin{Highlighting}[]
\OperatorTok{/}\ErrorTok{*}\NormalTok{using the logistic model to build propensity score model }\ControlFlowTok{for}\NormalTok{ having autologous }\KeywordTok{blood}\NormalTok{ (variable}\OperatorTok{:}\StringTok{ }\NormalTok{autoblood) }
\OperatorTok{/}\ErrorTok{*}\NormalTok{(autologous blood is done before cardiac surgery to collect the patients own blood); }

\OperatorTok{*}\StringTok{ }\NormalTok{rename autoblood as the treatment group; }
\NormalTok{data dat; set dat; group=AutoBlood; run;}
\OperatorTok{*}\StringTok{ }\NormalTok{assign the subject id; }
\NormalTok{data dat; set dat; subject_id=_n_;run;}


\OperatorTok{*}\StringTok{ }\NormalTok{the propensity score model }\ControlFlowTok{for}\NormalTok{ treatment group; }
\NormalTok{ proc logistic data=dat descending outest=betas; }
\NormalTok{   class  gender CAD_ Any_RF_Preop AcuteStroke PriorCardSx cardiogenicShock severe_AI }\OperatorTok{/}\NormalTok{param=ref ;}
\NormalTok{   model }\KeywordTok{AutoBlood}\NormalTok{ (}\DataTypeTok{event=}\StringTok{'1'}\NormalTok{)=}\StringTok{  }\NormalTok{age_at_operation  BMI RFHemoglobin_ gender AcuteStroke PriorCardSx cardiogenicShock    ; }
  \OperatorTok{*}\NormalTok{output the dataset }\StringTok{"predic"}\NormalTok{ which have predprob}\OperatorTok{:}\StringTok{ }\NormalTok{predprob is the probability scale, and xbeta is the linear predictor;}
\NormalTok{   output out=predic p=predprob xbeta=logit; }
\NormalTok{run;}

\OperatorTok{*}\StringTok{ }\ControlFlowTok{if}\NormalTok{ using the probability scale as the propensity score to do matching, then using predprob as the pscoreT}\OperatorTok{/}\NormalTok{C;}
\OperatorTok{*}\StringTok{ }\ControlFlowTok{if}\NormalTok{ using the logit scale as the propensity score }\ControlFlowTok{for}\NormalTok{ matching, then using xbeta as the pscoreT}\OperatorTok{/}\NormalTok{C; }
\NormalTok{data }\KeywordTok{treatment}\NormalTok{(}\DataTypeTok{rename =}\NormalTok{(}\DataTypeTok{subject_id=}\NormalTok{idT }\DataTypeTok{logit=}\NormalTok{ pscoreT  ) );set predic; where group=}\DecValTok{1}\NormalTok{ ; run;}
\NormalTok{data }\KeywordTok{control}\NormalTok{ (}\DataTypeTok{rename =}\NormalTok{(}\DataTypeTok{subject_id=}\NormalTok{idC }\DataTypeTok{logit=}\NormalTok{ pscoreC) ); set predic; where group=}\DecValTok{0}\NormalTok{; run;}

\OperatorTok{*}\StringTok{ }\NormalTok{using the propensity score matching macro; }
\NormalTok{%include }\StringTok{'xx\textbackslash{}propensity score matching macro\textbackslash{}psmatching.sas'}\NormalTok{;}

\NormalTok{proc means data=predic; var logit; output out=}\KeywordTok{stddata}\NormalTok{ (}\DataTypeTok{keep =}\NormalTok{std) std=std;run; }

\OperatorTok{*}\StringTok{ }\NormalTok{using the }\FloatTok{0.1}\NormalTok{ standardard deviation of the propensity score as the caliper }\ControlFlowTok{for}\NormalTok{ finding matching; }
\NormalTok{data stddata; set stddata; caliper=}\FloatTok{0.1}\OperatorTok{*}\NormalTok{std;run;  }


\OperatorTok{*}\NormalTok{one}\OperatorTok{-}\NormalTok{to}\OperatorTok{-}\NormalTok{one matching, no replacement, caliper matching; }
\NormalTok{%}\KeywordTok{PSMatching}\NormalTok{(}\DataTypeTok{datatreatment=}\NormalTok{ treatment, }\DataTypeTok{datacontrol=}\NormalTok{ control, }\DataTypeTok{method=}\NormalTok{ CALIPER,}
      \DataTypeTok{numberofcontrols=}\DecValTok{1}\NormalTok{, }\DataTypeTok{caliper=}\FloatTok{0.0844}\NormalTok{, }\DataTypeTok{replacement=}\NormalTok{ no, }\DataTypeTok{out=}\NormalTok{ caliper_matches2);  }


\NormalTok{data }\KeywordTok{treatment2}\NormalTok{ (}\DataTypeTok{keep=}\NormalTok{idT  }\DataTypeTok{rename=}\NormalTok{(}\DataTypeTok{idT=}\NormalTok{MatchedToTreatID) ); set treatment;run;}
\NormalTok{data }\KeywordTok{control2}\NormalTok{ (}\DataTypeTok{keep=}\NormalTok{idC  }\DataTypeTok{rename=}\NormalTok{(}\DataTypeTok{idC=}\NormalTok{IdSelectedControl  )); set control;run;}

\NormalTok{proc sort data=caliper_matches2 nodupkey out=test2; by MatchedToTreatID;run;   }

\OperatorTok{*}\StringTok{ }\NormalTok{create the data }\ControlFlowTok{for}\NormalTok{ matching sample; }
\NormalTok{data caliper_pairs;}
\NormalTok{    set  caliper_matches2;}
\NormalTok{    subject_id =}\StringTok{ }\NormalTok{IdSelectedControl; pscore =}\StringTok{ }\NormalTok{PScoreControl; pair =}\StringTok{ }\NormalTok{_N_;}
\NormalTok{    output;}
\NormalTok{    subject_id =}\StringTok{ }\NormalTok{MatchedToTreatID; pscore =}\StringTok{ }\NormalTok{PScoreTreat; pair =}\StringTok{ }\NormalTok{_N_;}
\NormalTok{    output;}
\NormalTok{    keep subject_id  pscore pair;run;}

\OperatorTok{*}\StringTok{ }\NormalTok{merging the dataset with the original sample to have the other covariates; }
\NormalTok{proc sort data=dat; by subject_id;run;}
\NormalTok{proc sort data=caliper_pairs; by subject_id;run;}

\NormalTok{data merge_pairs; }
\NormalTok{   merge }\KeywordTok{caliper_pairs}\NormalTok{ (}\DataTypeTok{in=}\NormalTok{a) }\KeywordTok{dat}\NormalTok{ (}\DataTypeTok{in=}\NormalTok{b); }
\NormalTok{   by subject_id;}
   \ControlFlowTok{if}\NormalTok{ a=}\DecValTok{1}\NormalTok{ and b=}\DecValTok{1}\NormalTok{;run;}


\OperatorTok{*}\NormalTok{check balance }\ControlFlowTok{for}\NormalTok{ continuous variables by plotting the empirical distribution between treatment and control groups ; }

\NormalTok{proc npar1way d edf plots=edfplot data=merge_pairs  scores=data;}
\NormalTok{class group;}
\NormalTok{var pscore  age_at_operation  BMI RFHemoglobin_  ;}
\NormalTok{run;}


\NormalTok{proc sort data=merge_pairs; by group;run;}
\NormalTok{proc means  data=merge_pairs mean median p25 p75; }
\NormalTok{var age_ pre_op_creatinine ef__pre_; }
\NormalTok{class group;}
\NormalTok{run;}


\OperatorTok{*}\StringTok{ }\NormalTok{MACRO}\OperatorTok{:}\StringTok{ }\NormalTok{use standardized difference }\ControlFlowTok{for}\NormalTok{ checking balance }\ControlFlowTok{for}\NormalTok{ binary variables;}
\NormalTok{%macro }\KeywordTok{binary}\NormalTok{ (}\DataTypeTok{var=}\NormalTok{, }\DataTypeTok{label=}\NormalTok{);}
\NormalTok{proc means  data=merge_pairs mean noprint; }
\NormalTok{var }\OperatorTok{&}\NormalTok{var; }
\NormalTok{by group;}
\NormalTok{output out=}\KeywordTok{outmean}\NormalTok{ (}\DataTypeTok{keep=}\NormalTok{group mean ) mean=mean ;}
\NormalTok{run;}

\NormalTok{data del0; set outmean; }\ControlFlowTok{if}\NormalTok{ group=}\DecValTok{0}\NormalTok{; mean_}\DecValTok{0}\NormalTok{ =mean;  keep mean_}\DecValTok{0}\NormalTok{;run;}
\NormalTok{data del1; set outmean; }\ControlFlowTok{if}\NormalTok{ group=}\DecValTok{1}\NormalTok{; mean_}\DecValTok{1}\NormalTok{ =mean;  keep mean_}\DecValTok{1}\NormalTok{;run;}


\NormalTok{data newdata; }
\NormalTok{  length label }\OperatorTok{$}\DecValTok{25}\NormalTok{;}
\NormalTok{  merge del0 del1;}
\NormalTok{  d=}\StringTok{ }\NormalTok{(mean_}\DecValTok{1}\OperatorTok{-}\NormalTok{mean_}\DecValTok{0}\NormalTok{)}\OperatorTok{/}\KeywordTok{sqrt}\NormalTok{ ((mean_}\DecValTok{1}\OperatorTok{*}\NormalTok{(}\DecValTok{1}\OperatorTok{-}\NormalTok{mean_}\DecValTok{1}\NormalTok{)}\OperatorTok{+}\NormalTok{mean_}\DecValTok{0}\OperatorTok{*}\NormalTok{(}\DecValTok{1}\OperatorTok{-}\NormalTok{mean_}\DecValTok{0}\NormalTok{))}\OperatorTok{/}\DecValTok{2}\NormalTok{);}
\NormalTok{  d=}\KeywordTok{round}\NormalTok{(}\KeywordTok{abs}\NormalTok{(d),}\FloatTok{0.0001}\NormalTok{);}
\NormalTok{  label=}\ErrorTok{&}\NormalTok{label;}
\NormalTok{  keep d label;}
\NormalTok{  run;}

\NormalTok{  proc append data=newdata base=}\KeywordTok{standiff}\NormalTok{ (}\DataTypeTok{keep=}\NormalTok{d label) force;run;}
\NormalTok{  %mend binary;}
\NormalTok{proc sort data=merge_pairs; by group;run;}


\OperatorTok{*}\StringTok{ }\NormalTok{make sure the variables were coded as numeric variables; }
\NormalTok{data merge_pairs; set merge_pairs; }\ControlFlowTok{if}\NormalTok{ gender=}\StringTok{"1"}\NormalTok{ then gender_=}\DecValTok{1}\NormalTok{; }\ControlFlowTok{else}\NormalTok{ gender_=}\DecValTok{0}\NormalTok{;run;}

\NormalTok\KeywordTok{binary}\NormalTok{ (}\DataTypeTok{var=}\NormalTok{CAD_, }\DataTypeTok{label=}\StringTok{"CAD_"}\NormalTok{)}
\NormalTok{%}\KeywordTok{binary}\NormalTok{ (}\DataTypeTok{var=}\NormalTok{severe_AI, }\DataTypeTok{label=}\StringTok{"severe_AI"}\NormalTok{)}

\NormalTok{proc print data=standiff;run;}


\OperatorTok{*}\StringTok{ }\NormalTok{using the matching samples, look at outcome difference with the conditional logistic model; }
\NormalTok{proc logistic data=merge_pairs; }
\NormalTok{  class }\KeywordTok{group}\NormalTok{ (}\DataTypeTok{ref=}\StringTok{'0'}\NormalTok{);}
\NormalTok{  model }\KeywordTok{OperativeMortality}\NormalTok{ (}\DataTypeTok{event=}\StringTok{'1'}\NormalTok{)=group; }
\NormalTok{  strata pair;}
\NormalTok{  run;}


\OperatorTok{*}\StringTok{ }\NormalTok{using the matching samples, look at survival difference with cox model accounting }\ControlFlowTok{for}\NormalTok{ the pairs; }
\NormalTok{proc phreg data=merge_pairs  ; }
\NormalTok{class }\KeywordTok{group}\NormalTok{ (}\DataTypeTok{ref=}\StringTok{'0'}\NormalTok{) pair; }
\NormalTok{model time_since_surgery}\OperatorTok{*}\KeywordTok{death}\NormalTok{ (}\DecValTok{0}\NormalTok{)=group;}
\NormalTok{random pair;}
\NormalTok{hazardratio group}\OperatorTok{/}\NormalTok{diff=ref;}
\NormalTok{run;}
\end{Highlighting}
\end{Shaded}

The below is the propensity score matching MACRO :

\begin{Shaded}
\begin{Highlighting}[]
\CommentTok{#SAS Macro for obtaining propensity score matching samples;}


\OperatorTok{/}\ErrorTok{************************************************}\StringTok{ }
\StringTok{    }\NormalTok{PSmatching.sas   adapted from}
    
\NormalTok{    Paper }\DecValTok{185}\OperatorTok{-}\DecValTok{2007}\NormalTok{  SAS Global Forum }\DecValTok{2007}
\NormalTok{    Local and Global Optimal Propensity Score Matching}
\NormalTok{    Marcelo Coca}\OperatorTok{-}\NormalTok{Perraillon}
\NormalTok{    Health Care Policy Department, Harvard Medical School, Boston, MA}
    
    \OperatorTok{-------------------------------}
\StringTok{ }\NormalTok{Treatment and Control observations must be }\ControlFlowTok{in}\NormalTok{ separate datasets such that }
\NormalTok{    Control data includes}\OperatorTok{:}\StringTok{ }\NormalTok{idC =}\StringTok{  }\NormalTok{subject_id, pscoreC =}\StringTok{ }\NormalTok{propensity score}
\NormalTok{    Treatment data includes}\OperatorTok{:}\StringTok{ }\NormalTok{idT, pscoreT}
\NormalTok{    id must be numeric  }
    
\NormalTok{    method =}\StringTok{ }\KeywordTok{NN}\NormalTok{ (nearest neighbor), caliper, or radius }
    
\NormalTok{    caliper value =}\StringTok{ }\NormalTok{max }\ControlFlowTok{for}\NormalTok{ matching}
    
\NormalTok{    replacement =}\StringTok{ }\NormalTok{yes}\OperatorTok{/}\NormalTok{no  whether controls can be matched to more than one case}
    
\NormalTok{    out =}\StringTok{ }\NormalTok{output data set name}
    
\NormalTok{    example call}\OperatorTok{:}
\StringTok{    }
\StringTok{     }\NormalTok{%}\KeywordTok{PSMatching}\NormalTok{(}\DataTypeTok{datatreatment=}\NormalTok{ T, }\DataTypeTok{datacontrol=}\NormalTok{ C, }\DataTypeTok{method=}\NormalTok{ NN,}
      \DataTypeTok{numberofcontrols=} \DecValTok{1}\NormalTok{, }\DataTypeTok{caliper=}\NormalTok{, }\DataTypeTok{replacement=}\NormalTok{ no, }\DataTypeTok{out=}\NormalTok{ matches);}
      
\NormalTok{ Output format}\OperatorTok{:}
\StringTok{           }\NormalTok{Id                  Matched}
\NormalTok{       Selected     PScore       To       PScore}
\NormalTok{Obs     Control    Control    TreatID     Treat}
  \DecValTok{1}      \DecValTok{18628}     \FloatTok{0.39192}     \DecValTok{16143}     \FloatTok{0.39192}
  \DecValTok{2}      \DecValTok{18505}     \FloatTok{0.23029}     \DecValTok{16158}     \FloatTok{0.23002}
  \DecValTok{3}      \DecValTok{15589}     \FloatTok{0.29260}     \DecValTok{16112}     \FloatTok{0.29260}

\NormalTok{All other variables discarded.   Reformat }\ControlFlowTok{for}\NormalTok{ merge on subject_id with original data}\OperatorTok{:}

\StringTok{  }\NormalTok{data pairs;}
\NormalTok{    set  matches;}
\NormalTok{    subject_id =}\StringTok{ }\NormalTok{IdSelectedControl; pscore =}\StringTok{ }\NormalTok{PScoreControl; pair =}\StringTok{ }\NormalTok{_N_;}
\NormalTok{    output;}
\NormalTok{    subject_id =}\StringTok{ }\NormalTok{MatchedToTreatID; pscore =}\StringTok{ }\NormalTok{PScoreTreat; pair =}\StringTok{ }\NormalTok{_N_;}
\NormalTok{    output;}
\NormalTok{    keep subject_id pscore pair;}


\OperatorTok{**}\ErrorTok{**********************************************/}

\NormalTok{%macro }\KeywordTok{PSMatching}\NormalTok{(}\DataTypeTok{datatreatment=}\NormalTok{, }\DataTypeTok{datacontrol=}\NormalTok{, }\DataTypeTok{method=}\NormalTok{, }\DataTypeTok{numberofcontrols=}\NormalTok{, }\DataTypeTok{caliper=}\NormalTok{,}
 \DataTypeTok{replacement=}\NormalTok{, }\DataTypeTok{out=}\NormalTok{);}

\OperatorTok{/}\ErrorTok{*}\StringTok{ }\NormalTok{Create copies of the treated units }\ControlFlowTok{if}\NormalTok{ N }\OperatorTok{>}\StringTok{ }\DecValTok{1} \OperatorTok{*}\ErrorTok{/}\NormalTok{;}
\NormalTok{ data }\KeywordTok{_Treatment0}\NormalTok{(}\DataTypeTok{drop=}\NormalTok{ i);}
\NormalTok{  set }\OperatorTok{&}\NormalTok{datatreatment;}
\NormalTok{  do i=}\StringTok{ }\DecValTok{1}\NormalTok{ to }\OperatorTok{&}\NormalTok{numberofcontrols;}
\NormalTok{  RandomNumber=}\StringTok{ }\KeywordTok{ranuni}\NormalTok{(}\DecValTok{12345}\NormalTok{);}
\NormalTok{output;}
\NormalTok{end;}
\NormalTok{run;}
\OperatorTok{/}\ErrorTok{*}\StringTok{ }\NormalTok{Randomly sort both datasets }\OperatorTok{*}\ErrorTok{/}
\NormalTok{proc sort data=}\StringTok{ }\NormalTok{_Treatment0 out=}\StringTok{ }\KeywordTok{_Treatment}\NormalTok{(}\DataTypeTok{drop=}\NormalTok{ RandomNumber);}
\NormalTok{by RandomNumber;}
\NormalTok{run;}
\NormalTok{data _Control0;}
\NormalTok{set }\OperatorTok{&}\NormalTok{datacontrol;}
\NormalTok{RandomNumber=}\StringTok{ }\KeywordTok{ranuni}\NormalTok{(}\DecValTok{45678}\NormalTok{);}
\NormalTok{run;}
\NormalTok{proc sort data=}\StringTok{ }\NormalTok{_Control0 out=}\StringTok{ }\KeywordTok{_Control}\NormalTok{(}\DataTypeTok{drop=}\NormalTok{ RandomNumber);}
\NormalTok{by RandomNumber;}
\NormalTok{run;}

\NormalTok{ data }\KeywordTok{Matched}\NormalTok{(}\DataTypeTok{keep =}\NormalTok{ IdSelectedControl PScoreControl MatchedToTreatID PScoreTreat);}
\NormalTok{  length pscoreC }\DecValTok{8}\NormalTok{;}
\NormalTok{  length idC }\DecValTok{8}\NormalTok{;}
\OperatorTok{/}\ErrorTok{*}\StringTok{ }\NormalTok{Load Control dataset into the hash object }\OperatorTok{*}\ErrorTok{/}
\StringTok{  }\ControlFlowTok{if}\NormalTok{ _N_=}\StringTok{ }\DecValTok{1}\NormalTok{ then do;}
\NormalTok{declare hash }\KeywordTok{h}\NormalTok{(dataset}\OperatorTok{:}\StringTok{ "_Control"}\NormalTok{, ordered}\OperatorTok{:}\StringTok{ 'no'}\NormalTok{);}
\NormalTok{declare hiter }\KeywordTok{iter}\NormalTok{(}\StringTok{'h'}\NormalTok{);}
\KeywordTok{h.defineKey}\NormalTok{(}\StringTok{'idC'}\NormalTok{);}
\KeywordTok{h.defineData}\NormalTok{(}\StringTok{'pscoreC'}\NormalTok{, }\StringTok{'idC'}\NormalTok{);}
\KeywordTok{h.defineDone}\NormalTok{();}
\NormalTok{call }\KeywordTok{missing}\NormalTok{(idC, pscoreC);}
\NormalTok{end;}
\OperatorTok{/}\ErrorTok{*}\StringTok{ }\NormalTok{Open the treatment }\OperatorTok{*}\ErrorTok{/}
\NormalTok{set _Treatment;}
\OperatorTok\KeywordTok{upcase}\NormalTok{(}\OperatorTok{&}\NormalTok{method) }\OperatorTok{~}\ErrorTok{=}\StringTok{ }\NormalTok{RADIUS }\OperatorTok\NormalTok{do;}
\NormalTok{retain BestDistance }\DecValTok{99}\NormalTok{;}
\NormalTok{%end;}
\OperatorTok{/}\ErrorTok{*}\StringTok{ }\NormalTok{Iterate over the hash }\OperatorTok{*}\ErrorTok{/}
\NormalTok{rc=}\StringTok{ }\KeywordTok{iter.first}\NormalTok{();}
\ControlFlowTok{if}\NormalTok{ (}\DataTypeTok{rc=}\DecValTok{0}\NormalTok{) then BestDistance=}\StringTok{ }\DecValTok{99}\NormalTok{;}
\NormalTok{do }\ControlFlowTok{while}\NormalTok{ (}\DataTypeTok{rc =} \DecValTok{0}\NormalTok{);}
\OperatorTok{/}\ErrorTok{*}\StringTok{ }\NormalTok{Caliper }\OperatorTok{*}\ErrorTok{/}
\OperatorTok\KeywordTok{upcase}\NormalTok{(}\OperatorTok{&}\NormalTok{method) =}\StringTok{ }\NormalTok{CALIPER }\OperatorTok\NormalTok{do;}
\ControlFlowTok{if}\NormalTok{ (pscoreT }\OperatorTok{-}\StringTok{ }\ErrorTok{&}\NormalTok{caliper) }\OperatorTok{<=}\StringTok{ }\NormalTok{pscoreC }\OperatorTok{<=}\StringTok{ }\NormalTok{(pscoreT }\OperatorTok{+}\StringTok{ }\ErrorTok{&}\NormalTok{caliper) then do;}
\NormalTok{ScoreDistance =}\StringTok{ }\KeywordTok{abs}\NormalTok{(pscoreT }\OperatorTok{-}\StringTok{ }\NormalTok{pscoreC);}
\ControlFlowTok{if}\NormalTok{ ScoreDistance }\OperatorTok{<}\StringTok{ }\NormalTok{BestDistance then do;}
\NormalTok{BestDistance =}\StringTok{ }\NormalTok{ScoreDistance;}
\NormalTok{IdSelectedControl =}\StringTok{ }\NormalTok{idC;}
\NormalTok{PScoreControl =}\StringTok{  }\NormalTok{pscoreC;}
\NormalTok{MatchedToTreatID =}\StringTok{ }\NormalTok{idT;}
\NormalTok{PScoreTreat =}\StringTok{ }\NormalTok{pscoreT;}
\NormalTok{end;}
\NormalTok{end;}
\NormalTok{%end;}
\OperatorTok{/}\ErrorTok{*}\StringTok{ }\NormalTok{NN }\OperatorTok{*}\ErrorTok{/}
\OperatorTok\KeywordTok{upcase}\NormalTok{(}\OperatorTok{&}\NormalTok{method) =}\StringTok{ }\NormalTok{NN }\OperatorTok\NormalTok{do;}
\NormalTok{ScoreDistance =}\StringTok{ }\KeywordTok{abs}\NormalTok{(pscoreT }\OperatorTok{-}\StringTok{ }\NormalTok{pscoreC);}
\ControlFlowTok{if}\NormalTok{ ScoreDistance }\OperatorTok{<}\StringTok{ }\NormalTok{BestDistance then do;}
\NormalTok{BestDistance =}\StringTok{ }\NormalTok{ScoreDistance;}
\NormalTok{IdSelectedControl =}\StringTok{ }\NormalTok{idC;}
\NormalTok{PScoreControl =}\StringTok{  }\NormalTok{pscoreC;}
\NormalTok{MatchedToTreatID =}\StringTok{ }\NormalTok{idT;}
\NormalTok{PScoreTreat =}\StringTok{ }\NormalTok{pscoreT;}
\NormalTok{end;}
\NormalTok{%end;}

\OperatorTok\KeywordTok{upcase}\NormalTok{(}\OperatorTok{&}\NormalTok{method) =}\StringTok{ }\NormalTok{NN or }\OperatorTok\NormalTok{then %do;}
\NormalTok{rc =}\StringTok{ }\KeywordTok{iter.next}\NormalTok{();}
\OperatorTok{/}\ErrorTok{*}\StringTok{ }\NormalTok{Output the best control and remove it }\OperatorTok{*}\ErrorTok{/}
\ControlFlowTok{if}\NormalTok{ (rc }\OperatorTok{~}\ErrorTok{=}\StringTok{ }\DecValTok{0}\NormalTok{) and BestDistance }\OperatorTok{~}\ErrorTok{=}\DecValTok{99}\NormalTok{ then do;}
\NormalTok{output;}
\OperatorTok\KeywordTok{upcase}\NormalTok{(}\OperatorTok{&}\NormalTok{replacement) =}\StringTok{ }\NormalTok{NO }\OperatorTok\NormalTok{do;}
\NormalTok{rc1 =}\StringTok{ }\KeywordTok{h.remove}\NormalTok{(key}\OperatorTok{:}\StringTok{ }\NormalTok{IdSelectedControl);}
\NormalTok{%end;}
\NormalTok{end;}
\NormalTok{%end;}
\OperatorTok{/}\ErrorTok{*}\StringTok{ }\NormalTok{Radius }\OperatorTok{*}\ErrorTok{/}
\OperatorTok\KeywordTok{upcase}\NormalTok{(}\OperatorTok{&}\NormalTok{method) =}\StringTok{ }\NormalTok{RADIUS }\OperatorTok\NormalTok{do;}
\ControlFlowTok{if}\NormalTok{ (pscoreT }\OperatorTok{-}\StringTok{ }\ErrorTok{&}\NormalTok{caliper) }\OperatorTok{<=}\StringTok{ }\NormalTok{pscoreC }\OperatorTok{<=}\StringTok{ }\NormalTok{(pscoreT }\OperatorTok{+}\StringTok{ }\ErrorTok{&}\NormalTok{caliper) then do;}
\NormalTok{IdSelectedControl =}\StringTok{ }\NormalTok{idC;}
\NormalTok{PScoreControl =}\StringTok{  }\NormalTok{pscoreC;}
\NormalTok{MatchedToTreatID =}\StringTok{ }\NormalTok{idT;}
\NormalTok{PScoreTreat =}\StringTok{ }\NormalTok{pscoreT;}
\NormalTok{output;}
\NormalTok{end;}
\NormalTok{rc =}\StringTok{ }\KeywordTok{iter.next}\NormalTok{();}
\NormalTok{%end;}
\NormalTok{end;}
\NormalTok{run;}
\OperatorTok{/}\ErrorTok{*}\StringTok{ }\NormalTok{Delete temporary tables. Quote }\ControlFlowTok{for}\NormalTok{ debugging }\OperatorTok{*}\ErrorTok{/}
\NormalTok{proc datasets;}
\NormalTok{delete _}\OperatorTok{:}\NormalTok{(}\DataTypeTok{gennum=}\NormalTok{all);}

\NormalTok{run;}
\NormalTok{ data }\OperatorTok{&}\NormalTok{out;}
\NormalTok{   set Matched;}
\NormalTok{ run;}
\NormalTok{%mend PSMatching;}
\end{Highlighting}
\end{Shaded}

\end{document}
